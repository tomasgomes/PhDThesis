% ************************** Thesis Acknowledgements **************************

\begin{acknowledgements}      


I would need an additional very long chapter to fully and fairly acknowledge each and every person that contributed to this endeavour.

I would like to acknowledge my supervisor, Sarah Teichmann, for the opportunity to pursue this PhD. It has been a long and winding journey, and I could not have done it without her guidance. Sarah was always ready to provide valuable, original insights on the problems and questions I had, teaching me so much more than I have ever expected. I fell truly fortunate to have her as a mentor that I can rely on.

Besides being a brilliant scientist, one of Sarah's greatest achievements has been to put together and maintain a highly collaborative lab, full of outstanding people always ready to inspire and help. It is hard to find someone who has not had an impact on my progress and research, but I would like to name a few. To Ricardo Miragaia, the other half of the Portuguese dynamic duo, for the relentless understanding and support provided. Through good and bad times, we had a lot of fun together, and I would not have it any other way. I also want to acknowledge my "Jedi master" Roser Vento, for the interesting, deep discussions and crucial encouragement throughout my PhD. I am lucky to have such a brilliant and dedicated friend. To Tzachi Hagai, who is a constant source of inspiration, enthusiasm and skepticism. He has taught me to question everything, and through better knowing my limitations make the most of them. Valentine Svensson was one of the people that I learned the most from. Likely the most talented young scientist I have ever met, he was crucial in my understanding of single-cell computational methods, and always a fun company to be around. Raghd Rostom, my "PhD sister", whose funny antics and incisive comments have helped keep my head grounded. To Kylie James, for the uplifting conversations about life, they have always left me with a feeling of hope for the future. And to Rasa Elmentaite, for her inspiring dedication and contagious desire to learn and ask questions.

I am greatful to the whole ENLIGH-TEN consortium, which funded my PhD and provided me with the opportunity to meet several other early stage researchers and their supervisors. The support from this network has been crucial for my research, and gave me a true sense of what it is like to do science across borders.

I could not have started this PhD without the assitance of my former advisors Maria do Carmo Fonseca and Ana Rita Grosso, as well as collaborators Nick Proudfoot and Taka Nojima. They were crucial in starting my career in bioinformatics and genomics, and are overall great scientists and supportive mentors.

I would also like to thank all my friends, in particular to Tiago Pires, who despite the distance always gave me an escape from the hectic PhD life. I would also like to thank Mariana, Elsa, Gonçalo, and Joana for the fun times we had in Cambridge and with the Cambridge University Portuguese Speakers' Society. And to Gianmarco Raddi, who I could always count on to keep my spirits up.

I am deeply grateful to all my family, my father José Carlos, my mother Ana Luísa, my brother Miguel, my aunt Fernanda, and my grandmother Maria do Carmo.They have been relentless in their assistance and understanding, and it is thanks to them that I got the opportunity to work on what I love. I also want to posthumously thank my grandfather Fernando for all the inspiration and all he taught me.

Finally, and most importantly, I want to thank my partner Hajrabibi Ali. I am deeply indebted to her for her encouragement, patience and self-sacrifice, that ultimately help me pursue this degree. I am hoping that I can ever repay her for her time and help in keeping me focused and reminding me of the greater picture.

\end{acknowledgements}
