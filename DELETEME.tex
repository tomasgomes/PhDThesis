

% CHANGE below is still some example text that might later be useful
The most famous equation in the world: $E^2 = (m_0c^2)^2 + (pc)^2$, which is 
known as the \textbf{energy-mass-momentum} relation as an in-line equation.

A {\em \LaTeX{} class file}\index{\LaTeX{} class file@LaTeX class file} is a file, which holds style information for a particular \LaTeX{}.

\begin{align}
CIF: \hspace*{5mm}F_0^j(a) = \frac{1}{2\pi \iota} \oint_{\gamma} \frac{F_0^j(z)}{z - a} dz
\end{align}

\nomenclature[z-cif]{$CIF$}{Cauchy's Integral Formula}                                % first letter Z is for Acronyms 
\nomenclature[a-F]{$F$}{complex function}                                                   % first letter A is for Roman symbols
\nomenclature[g-p]{$\pi$}{ $\simeq 3.14\ldots$}                                             % first letter G is for Greek Symbols
\nomenclature[g-i]{$\iota$}{unit imaginary number $\sqrt{-1}$}                      % first letter G is for Greek Symbols
\nomenclature[g-g]{$\gamma$}{a simply closed curve on a complex plane}  % first letter G is for Greek Symbols
\nomenclature[x-i]{$\oint_\gamma$}{integration around a curve $\gamma$} % first letter X is for Other Symbols
\nomenclature[r-j]{$j$}{superscript index}                                                       % first letter R is for superscripts
\nomenclature[s-0]{$0$}{subscript index}                                                        % first letter S is for subscripts


% Uncomment this line, when you have siunitx package loaded.
The SI Units for dynamic viscosity is \si{\newton\second\per\metre\squared}.
I'm going to randomly include a picture Figure~\ref{fig:minion}.


If you have trouble viewing this document contact Krishna at: \href{mailto:kks32@cam.ac.uk}{kks32@cam.ac.uk} or raise an issue at \url{https://github.com/kks32/phd-thesis-template/}





\section*{Enumeration}
Lorem ipsum dolor sit amet, consectetur adipiscing elit. Sed vitae laoreet lectus. Donec lacus quam, malesuada ut erat vel, consectetur eleifend tellus. Aliquam non feugiat lacus. Interdum et malesuada fames ac ante ipsum primis in faucibus. Quisque a dolor sit amet dui malesuada malesuada id ac metus. Phasellus posuere egestas mauris, sed porta arcu vulputate ut. Donec arcu erat, ultrices et nisl ut, ultricies facilisis urna. Quisque iaculis, lorem non maximus pretium, dui eros auctor quam, sed sodales libero felis vel orci. Aliquam neque nunc, elementum id accumsan eu, varius eu enim. Aliquam blandit ante et ligula tempor pharetra. Donec molestie porttitor commodo. Integer rutrum turpis ac erat tristique cursus. Sed venenatis urna vel tempus venenatis. Nam eu rhoncus eros, et condimentum elit. Quisque risus turpis, aliquam eget euismod id, gravida in odio. Nunc elementum nibh risus, ut faucibus mauris molestie eu.
 Vivamus quis nunc nec nisl vulputate fringilla. Duis tempus libero ac justo laoreet tincidunt. Fusce sagittis gravida magna, pharetra venenatis mauris semper at. Nullam eleifend felis a elementum sagittis. In vel turpis eu metus euismod tempus eget sit amet tortor. Donec eu rhoncus libero, quis iaculis lectus. Aliquam erat volutpat. Proin id ullamcorper tortor. Fusce vestibulum a enim non volutpat. Nam ut interdum nulla. Proin lacinia felis malesuada arcu aliquet fringilla. Aliquam condimentum, tellus eget maximus porttitor, quam sem luctus massa, eu fermentum arcu diam ac massa. Praesent ut quam id leo molestie rhoncus. Praesent nec odio eget turpis bibendum eleifend non sit amet mi. Curabitur placerat finibus velit, eu ultricies risus imperdiet ut. Suspendisse lorem orci, luctus porta eros a, commodo maximus nisi.

Nunc et dolor diam. Phasellus eu justo vitae diam vehicula tristique. Vestibulum vulputate cursus turpis nec commodo. Etiam elementum sit amet erat et pellentesque. In eu augue sed tortor mollis tincidunt. Mauris eros dui, sagittis vestibulum vestibulum vitae, molestie a velit. Donec non felis ut velit aliquam convallis sit amet sit amet velit. Aliquam vulputate, elit in lacinia lacinia, odio lacus consectetur quam, sit amet facilisis mi justo id magna. Curabitur aliquet pulvinar eros. Cras metus enim, tristique ut magna a, interdum egestas nibh. Aenean lorem odio, varius a sollicitudin non, cursus a odio. Vestibulum ante ipsum primis in faucibus orci luctus et ultrices posuere cubilia Curae; 
\begin{enumerate}
\item The first topic is dull
\item The second topic is duller
\begin{enumerate}
\item The first subtopic is silly
\item The second subtopic is stupid
\end{enumerate}
\item The third topic is the dullest
\end{enumerate}
Morbi bibendum est aliquam, hendrerit dolor ac, pretium sem. Nunc molestie, dui in euismod finibus, nunc enim viverra enim, eu mattis mi metus id libero. Cras sed accumsan justo, ut volutpat ipsum. Nam faucibus auctor molestie. Morbi sit amet eros a justo pretium aliquet. Maecenas tempor risus sit amet tincidunt tincidunt. Curabitur dapibus gravida gravida. Vivamus porta ullamcorper nisi eu molestie. Ut pretium nisl eu facilisis tempor. Nulla rutrum tincidunt justo, id placerat lacus laoreet et. Sed cursus lobortis vehicula. Donec sed tortor et est cursus pellentesque sit amet sed velit. Proin efficitur posuere felis, porta auctor nunc. Etiam non porta risus. Pellentesque lacinia eros at ante iaculis, sed aliquet ipsum volutpat. Suspendisse potenti.

Ut ultrices lectus sed sagittis varius. Nulla facilisi. Nullam tortor sem, placerat nec condimentum eu, tristique eget ex. Nullam pretium tellus ut nibh accumsan elementum. Aliquam posuere gravida tellus, id imperdiet nulla rutrum imperdiet. Nulla pretium ullamcorper quam, non iaculis orci consectetur eget. Curabitur non laoreet nisl. Maecenas lacinia, lorem vel tincidunt cursus, odio lorem aliquet est, gravida auctor arcu urna id enim. Morbi accumsan bibendum ipsum, ut maximus dui placerat vitae. Nullam pretium ac tortor nec venenatis. Nunc non aliquet neque. 

\section*{Itemize}
\begin{itemize}
\item The first topic is dull
\item The second topic is duller
\begin{itemize}
\item The first subtopic is silly
\item The second subtopic is stupid
\end{itemize}
\item The third topic is the dullest
\end{itemize}

\section*{Description}
\begin{description}
\item[The first topic] is dull
\item[The second topic] is duller
\begin{description}
\item[The first subtopic] is silly
\item[The second subtopic] is stupid
\end{description}
\item[The third topic] is the dullest
\end{description}


\clearpage

\tochide\section{Hidden section}
\textbf{Lorem ipsum dolor sit amet}, \textit{consectetur adipiscing elit}. In magna nisi, aliquam id blandit id, congue ac est. Fusce porta consequat leo. Proin feugiat at felis vel consectetur. Ut tempus ipsum sit amet congue posuere. Nulla varius rutrum quam. Donec sed purus luctus, faucibus velit id, ultrices sapien. Cras diam purus, tincidunt eget tristique ut, egestas quis nulla. Curabitur vel iaculis lectus. Nunc nulla urna, ultrices et eleifend in, accumsan ut erat. In ut ante leo. Aenean a lacinia nisl, sit amet ullamcorper dolor. Maecenas blandit, tortor ut scelerisque congue, velit diam volutpat metus, sed vestibulum eros justo ut nulla. Etiam nec ipsum non enim luctus porta in in massa. Cras arcu urna, malesuada ut tellus ut, pellentesque mollis risus.Morbi vel tortor imperdiet arcu auctor mattis sit amet eu nisi. Nulla gravida urna vel nisl egestas varius. Aliquam posuere ante quis malesuada dignissim. Mauris ultrices tristique eros, a dignissim nisl iaculis nec. Praesent dapibus tincidunt mauris nec tempor. Curabitur et consequat nisi. Quisque viverra egestas risus, ut sodales enim blandit at. Mauris quis odio nulla. Cras euismod turpis magna, in facilisis diam congue non. Mauris faucibus nisl a orci dictum, et tempus mi cursus.

Etiam elementum tristique lacus, sit amet eleifend nibh eleifend sed \footnote{My footnote goes blah blah blah! \dots}. Maecenas dapibu augue ut urna malesuada, non tempor nibh mollis. Donec sed sem sollicitudin, convallis velit aliquam, tincidunt diam. In eu venenatis lorem. Aliquam non augue porttitor tellus faucibus porta et nec ante. Proin sodales, libero vitae commodo sodales, dolor nisi cursus magna, non tincidunt ipsum nibh eget purus. Nam rutrum tincidunt arcu, tincidunt vulputate mi sagittis id. Proin et nisi nec orci tincidunt auctor et porta elit. Praesent eu dolor ac magna cursus euismod. Integer non dictum nunc.


\begin{landscape}

\section*{Subplots}
I can cite Wall-E (see Fig.~\ref{fig:WallE}) and Minions in despicable me (Fig.~\ref{fig:Minnion}) or I can cite the whole figure as Fig.~\ref{fig:animations}


\begin{figure}
  \centering
  \begin{subfigure}[b]{0.3\textwidth}
    \includegraphics[width=\textwidth]{TomandJerry}
    \caption{Tom and Jerry}
    \label{fig:TomJerry}   
  \end{subfigure}             
  \begin{subfigure}[b]{0.3\textwidth}
    \includegraphics[width=\textwidth]{WallE}
    \caption{Wall-E}
    \label{fig:WallE}
  \end{subfigure}             
  \begin{subfigure}[b]{0.3\textwidth}
    \includegraphics[width=\textwidth]{minion}
    \caption{Minions}
    \label{fig:Minnion}
  \end{subfigure}
  \caption{Best Animations}
  \label{fig:animations}
\end{figure}


\end{landscape}





And now I begin my third chapter here \dots

And now to cite some more people~\citet{Rea85,Ancey1996}

\subsection{First subsection in the first section}
\dots and some more 

\subsection{Second subsection in the first section}
\dots and some more \dots

\subsubsection{First subsub section in the second subsection}
\dots and some more in the first subsub section otherwise it all looks the same
doesn't it? well we can add some text to it \dots

\subsection{Third subsection in the first section}
\dots and some more \dots

\subsubsection{First subsub section in the third subsection}
\dots and some more in the first subsub section otherwise it all looks the same
doesn't it? well we can add some text to it and some more and some more and
some more and some more and some more and some more and some more \dots

\subsubsection{Second subsub section in the third subsection}
\dots and some more in the first subsub section otherwise it all looks the same
doesn't it? well we can add some text to it \dots

\section{Second section of the third chapter}
and here I write more \dots

\section{The layout of formal tables}
This section has been modified from ``Publication quality tables in \LaTeX*''
 by Simon Fear.

The layout of a table has been established over centuries of experience and 
should only be altered in extraordinary circumstances. 

When formatting a table, remember two simple guidelines at all times:

\begin{enumerate}
  \item Never, ever use vertical rules (lines).
  \item Never use double rules.
\end{enumerate}

These guidelines may seem extreme but I have
never found a good argument in favour of breaking them. For
example, if you feel that the information in the left half of
a table is so different from that on the right that it needs
to be separated by a vertical line, then you should use two
tables instead. Not everyone follows the second guideline:

There are three further guidelines worth mentioning here as they
are generally not known outside the circle of professional
typesetters and subeditors:

\begin{enumerate}\setcounter{enumi}{2}
  \item Put the units in the column heading (not in the body of
          the table).
  \item Always precede a decimal point by a digit; thus 0.1
      {\em not} just .1.
  \item Do not use `ditto' signs or any other such convention to
      repeat a previous value. In many circumstances a blank
      will serve just as well. If it won't, then repeat the value.
\end{enumerate}

A frequently seen mistake is to use `\textbackslash begin\{center\}' \dots `\textbackslash end\{center\}' inside a figure or table environment. This center environment can cause additional vertical space. If you want to avoid that just use `\textbackslash centering'


\begin{table}
\caption{A badly formatted table}
\centering
\label{table:bad_table}
\begin{tabular}{|l|c|c|c|c|}
\hline 
& \multicolumn{2}{c}{Species I} & \multicolumn{2}{c|}{Species II} \\ 
\hline
Dental measurement  & mean & SD  & mean & SD  \\ \hline 
\hline
I1MD & 6.23 & 0.91 & 5.2  & 0.7  \\
\hline 
I1LL & 7.48 & 0.56 & 8.7  & 0.71 \\
\hline 
I2MD & 3.99 & 0.63 & 4.22 & 0.54 \\
\hline 
I2LL & 6.81 & 0.02 & 6.66 & 0.01 \\
\hline 
CMD & 13.47 & 0.09 & 10.55 & 0.05 \\
\hline 
CBL & 11.88 & 0.05 & 13.11 & 0.04\\ 
\hline 
\end{tabular}
\end{table}

\begin{table}
\caption{A nice looking table}
\centering
\label{table:nice_table}
\begin{tabular}{l c c c c}
\hline 
\multirow{2}{*}{Dental measurement} & \multicolumn{2}{c}{Species I} & \multicolumn{2}{c}{Species II} \\ 
\cline{2-5}
  & mean & SD  & mean & SD  \\ 
\hline
I1MD & 6.23 & 0.91 & 5.2  & 0.7  \\

I1LL & 7.48 & 0.56 & 8.7  & 0.71 \\

I2MD & 3.99 & 0.63 & 4.22 & 0.54 \\

I2LL & 6.81 & 0.02 & 6.66 & 0.01 \\

CMD & 13.47 & 0.09 & 10.55 & 0.05 \\

CBL & 11.88 & 0.05 & 13.11 & 0.04\\ 
\hline 
\end{tabular}
\end{table}


\begin{table}
\caption{Even better looking table using booktabs}
\centering
\label{table:good_table}
\begin{tabular}{l c c c c}
\toprule
\multirow{2}{*}{Dental measurement} & \multicolumn{2}{c}{Species I} & \multicolumn{2}{c}{Species II} \\ 
\cmidrule{2-5}
  & mean & SD  & mean & SD  \\ 
\midrule
I1MD & 6.23 & 0.91 & 5.2  & 0.7  \\

I1LL & 7.48 & 0.56 & 8.7  & 0.71 \\

I2MD & 3.99 & 0.63 & 4.22 & 0.54 \\

I2LL & 6.81 & 0.02 & 6.66 & 0.01 \\

CMD & 13.47 & 0.09 & 10.55 & 0.05 \\

CBL & 11.88 & 0.05 & 13.11 & 0.04\\ 
\bottomrule
\end{tabular}
\end{table}
