% ************************** Thesis Abstract *****************************
% Use `abstract' as an option in the document class to print only the titlepage and the abstract.
\begin{abstract}
Cells are the building blocks of life, forming the vast diversity of tissues and organisms in Nature. Across these, common cellular morphologies and functions have been identified. High-throughput, multifactorial profiling of cells has grown exponentially in recent years with the advent of single-cell RNA-sequencing (scRNA-seq), increasingly unraveling cell diversity. Nonetheless, it is not yet known how different environments affect cellular phenotypes.

The work presented on this Thesis reports on the transcriptional variation of cell types across different tissues, by use of single-cell RNA-sequencing. This technology, developed in the last 10 years, has greatly impacted our ability to distinguish cellular heterogeneity by their gene expression in various tissues or conditions.

Chapter 1 outlines the impact o single-cell RNA-sequencing in cell biology. It presents the method as the natural progression of more low throughput or low resolution methods. The chapter then outlines how cellular heterogeneity can be deconstructed by analysing this type of genomics data. It then expands on how the data can be used to build useful models of cell type identity for automatic annotation, ultimately outlining the need to create a global cell type census of the whole organism. Such an endeavour can serve as a vital resource for automatic annotation, as well as to obtain the first cross-tissue integrative overview of cell identity.

The same chapter also delves into the topic of heterogeneity in immune cells. Due to the evolutionary pressure they are subject to and ubiquitous nature across the organism, these are some of the most diverse cell types in multicellular organisms. Thus, in Chapter 2, a deconstruction of T-regulatory cells' phenotypes in different mouse and human tissues is presented. These cells are known not just for their role in controlling the immune response, but also for other functions regulating the homeostasis of specific tissues. The chapter uses scRNA-seq data to further expand on how these cells adapt as they migrate between lymphoid and non-lymphoid tissues, and assesses the conservation of gene expression programmes for the same populations between mouse and human.

The creation of a global cell type reference is an endeavour that can facilitate analysis of new data, and reveal novel insights about cell and tissue biology. Several datasets have now been produced, and a method that can efficiently integrate them and prepare them for use as a reference is necessary. Chapter 3 details the development of such method, exploring its strengths and how it can be improved, in a mouse dataset. Chapter 4 then applies this pipeline to a collection of human data, and shows how cell types relate across tissues, as well as how the human reference can be used in a practical case.

Lastly, Chapter 5 summarises all chapters, providing an overview on how single-cell sequencing has changed what we know about tissue biology, and how listing cell types and compiling them as a functional reference can help future developments in life sciences.

\end{abstract}
